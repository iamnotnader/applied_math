%% LyX 2.0.3 created this file.  For more info, see http://www.lyx.org/.
%% Do not edit unless you really know what you are doing.
\documentclass[english]{article}
\usepackage[T1]{fontenc}
\usepackage[latin9]{inputenc}
\usepackage{geometry}
\geometry{verbose,tmargin=0.75in,bmargin=0.75in,lmargin=0.75in,rmargin=1in}
\usepackage{amsmath}
\makeatletter

%%%%%%%%%%%%%%%%%%%%%%%%%%%%%% LyX specific LaTeX commands.
%% Because html converters don't know tabularnewline
\providecommand{\tabularnewline}{\\}

\makeatother

\usepackage{babel}
\begin{document}
\thispagestyle{empty}

\begin{tabular*}{1\textwidth}{@{\extracolsep{\fill}}lr}
\textbf{ID:} 03 & \textbf{Collaborator \#1:} Last Name, First Name\tabularnewline
\textbf{Name:} Al-Naji, Nader & \textbf{Collaborator \#2:} Last Name, First Name\tabularnewline
\hline 
\end{tabular*}

\medskip{}


\begin{center}
\begin{Large}\textbf{Solution to HW 5, Problem 1}\end{Large}
\par\end{center}

\begin{center}
\begin{large}\textbf{COS 340 - Spring 2012}\end{large}
\par\end{center}


\bigskip{}

%begin sol
There are $b$ boys and $g$ girls arranged in a line. Let $n = b+g$ be the number of students. Let $s$ be the number of locations
where a boy and a girl stand next to each other. For example, for BBGGBBGGGB, the value of $s$ is 4 and $n$ is 10.
\newline
\newline
\textbf{1. Count the number of arrangements in which $s$ is even.}
\newline
\newline
Call a place where a boy and a girl stand next to each other in the line a ``shift.''  Using this language, it should be
clear that a line with an even number of shifts must start and end with a person of the same gender. To convince yourself of this,
simply consider that if the number of shifts is even, then, starting from the first person in line, every shift to a person of a 
gender different from that of the gender of the first person in line must be accompanied by a shift back to a person of the
same gender. So every shift to a different gender must be paired with a shift back in order for the number of shifts to be even.
It should also be stated that, by the same logic, that if the gender of the first person in line is the same as the gender of
the last person in line, then the number of shifts is even. Finally, note that every arrangement in which $s$ is even maps to
one and only one arrangement in which the end-persons have the same gender and vice versa. That is, it is not the case that 
an arrangement in which $s$ happens to be even counts as two separate arrangements in which the gender of the end-persons is the
same and vice versa. Having noted this, we can compute the number of arrangements in which
$s$ is even by computing the number of arrangements in which the gender of the end-persons is the same, since the size of both of these
sets must be the same (since they form a bijection). 
\newline\newline
The number of arrangements in which the end-persons have the same 
gender can be computed by first fixing the gender of the end-persons to be $B$ and permuting the inside $n-2$ persons and then fixing
the gender of the end-persons to be $G$ and then permuting the inside $n-2$ persons, and summing these two quantities. Assuming all persons
of the same gender are indistinguishable, we thus have that the number of arrangements is:
\newline\newline
${n-2 \choose b} + {n-2 \choose g}$.
\newline
\newline
However, because persons of the same gender are actually distinct, every arrangement must count for $b!g!$ different
arrangements since every permutation of the boys and girls while preserving gender location must be counted. Thus,
because arrangements in which the boys/girls are distinct form a $k-to-1$ mapping onto arrangements where the 
boys/girls are indistinguishable, where $k = b!g!$, we must multiply the number above by $b!g!$ and get:
\newline
\newline
$b!g![{n-2 \choose b} + {n-2 \choose g}]$
\newline
\newline
Finally, because this only holds for $n > 2$, we have to deal with the degenerate cases where $n=2$, $n = 1$, and $n = 0$. 
As the final answer we have, covering all these cases:
\newline
\newline
$$
s(b, g, n=b+g) =
\begin{cases}
b!g![{n-2 \choose b} + {n-2 \choose g}], & \text{if }n > 2 \\
2, & \text{if }n=2\text{ and }b \neq g \\
0, & \text{if }n=2\text{ and }b = g \\
1, & \text{if } n = 1 \text{ or } n=0 
\end{cases}
$$
\textbf{2. What is the expected value of $s$ if the students are permuted randomly?}
\newline\newline
Consider the gaps between each boy and girl in the line. Define $G_i$ to be an indicator random variable that is $1$ if 
gap $i$ is a shift and $0$ if gap $i$ is not a shift. There are $n-1$ such gaps in any given line of boys
and girls and, therefore $n-1$ such random variables. We want the expected number of shifts or, in other words,
the expected number of gaps that are $1$. But this is simply, using linearity of expectation:
\newline\newline
$E[s] = E[\sum\limits_{i=1}^{n-1} G_i] = \sum\limits_{i=1}^{n-1} E[G_i] = \sum\limits_{i=1}^{n-1} E[G_1] = (n-1)G_1$ since the $G_i$ are
identically distributed. 
\newline\newline
It should be clear that the $G_i$ are identically distributed since it doesn't make sense that one gap would be more likely to contain a shift
than any other. 
\newline\newline
So if we can figure out the distribution for $G_1$, we have our solution. To get the probability that $G_i$ is a shift, we simply sum 
the probabilities of two events: the event that a girl is on the left and a boy is on the right, and the event that a boy is on the
left and a girl is on the right. It should be clear that these are the two cases under which gap $i$ is a shift. Computing this, we get:
\newline
\newline
$P($space $i$ is a shift$) = P($boy on left, girl on right$) + P($girl on left, boy on right$)
\newline
= P($boy on left$)P($girl on left | boy on right$) + P($girl on left$)P($boy on right | girl on left$) $ by the chain rule$
\newline
= \frac{b}{n}\frac{g}{n-1} + \frac{g}{n}\frac{b}{n-1} = 2 \frac{bg}{n(n-1)}$.
\newline
\newline
Now that we have the probability that $G_i$ is a shift, we can finish our computation:
\newline
\newline
$E[s] = (n-1)G_1 = (n-1)2\frac{bg}{n(n-1)} = 2 \frac{bg}{n}$.
\newline
\newline

%end sol

\pagebreak{}

\thispagestyle{empty}

\begin{tabular*}{1\textwidth}{@{\extracolsep{\fill}}lr}
\textbf{ID:} 03 & \textbf{Collaborator \#1:} Last Name, First Name\tabularnewline
\textbf{Name:} Al-Naji, Nader & \textbf{Collaborator \#2:} Last Name, First Name\tabularnewline
\hline 
\end{tabular*}

\medskip{}


\begin{center}
\begin{Large}\textbf{Solution to HW 5, Problem 2}\end{Large}
\par\end{center}

\begin{center}
\begin{large}\textbf{COS 340 - Spring 2012}\end{large}
\par\end{center}


\bigskip{}

%begin sol
A box has $n$ balls each with a different color. We repeatedly draw a random ball from the box and return it. 
Let $R$ be the random variable that denotes the number of draws until the first time that some ball is picked twice. 
\newline
\newline
\textbf{1. Compute $P[R>k]$ for any integer $k \geq 0$.}
\newline
\newline
$P[R > k] = \prod\limits_{k=0}^{k-1} (n-k)/n$. We will reduce this to a closed form at the end but first let us prove that this
is true by induction.
\newline
\newline
$Q(k) = $ ``$P[R > k] = \prod\limits_{k=0}^{k-1} (n-k)/n$ for all $k \leq n$.''
\newline
\newline
\textbf{Base: $k = 1$}
\newline
\newline
For $k = 1$, we need to get past the first draw without picking a ball we've seen before. This happens with probability $1$
since we haven't seen any balls before, which agrees with the claim.
\newline
\newline
\textbf{Step: Assume true for $k$. Prove true for $k+1$.}
\newline
\newline
If we want $R > k+1$, then we need to get through $k$ choosings without picking a ball we've seen before, and then we need to pick
a ball we haven't seen before one more time. So $P[R > k+1] = P[R > k] \cdot (n - k)/n =  \prod\limits_{k=0}^{(k+1) - 1} (n-k)/n$, which agrees
with the claim.
\newline
\newline
Now that it is clear that $P[R > k] = \prod\limits_{k=0}^{k-1} (n-k)/n$, let us reduce this to a simpler expression.
\newline
\newline
$P[R > k] = \prod\limits_{k=0}^{k-1} (n-k)/n = \frac{(n)(n-1)...(n-k+1)}{n^k} = \frac{n!}{(n-k)! n^k} 
\newline
\newline
= {n \choose k} \frac{k!}{n^k}$
\newline
\newline
Finally, note that after $n$ draws, it is impossible to draw a ball we haven't seen before and $P[R > k] = 0$. So, as the final answer
we have:
\newline
\newline
$P[R > k] =
\begin{cases}
{n \choose k} \frac{k!}{n^k}, & \text{if } 0 \leq k \leq n \\
0, & \text{otherwise }
\end{cases}$
\newline
\newline
\newline
\textbf{2. Determine $E[R^2]$ within constant factors. Show your calculations.}
\newline
\newline
We will establish an upper bound and then establish a lower bound. We will start with the upper bound.
\newline
\newline
First, we need to derive an expression for $E[R^2]$ in terms of $P[R > k]$. To do this, let's take a look at what 
$E[R^2]$ actually is by definition:
\newline
\newline
$E[R^2] = \sum\limits_{k=0}^{\infty} P[R = k]k^2 = \sum\limits_{k=0}^{n} P[R = k] k^2$ since $P[R = k] = 0$ for $k > n$.
\newline
If we expand this out, we begin to notice a pattern:
\newline
\newline
$\sum\limits_{k=0}^{n} P[R=k]k^2 = $
\newline
$
P[R = 1] + P[R = 2] + P[R = 3] + P[R = 4] ... \newline
\mbox{~~~~~~~~~~~~}+ P[R = 2] + P[R = 3] + P[R = 4] ... \newline
\mbox{~~~~~~~~~~~~}+ P[R = 2] + P[R = 3] + P[R = 4] ... \newline
\mbox{~~~~~~~~~~~~}+ P[R = 2] + P[R = 3] + P[R = 4] ... \newline
\mbox{~~~~~~~~~~~~}\mbox{~~~~~~~~~~~~~~~~}+ P[R = 3] + P[R = 4] ... \newline
\mbox{~~~~~~~~~~~~}\mbox{~~~~~~~~~~~~~~~~}+ P[R = 3] + P[R = 4] ... \newline
\mbox{~~~~~~~~~~~~}\mbox{~~~~~~~~~~~~~~~~}+ P[R = 3] + P[R = 4] ... \newline
\mbox{~~~~~~~~~~~~}\mbox{~~~~~~~~~~~~~~~~}+ P[R = 3] + P[R = 4] ... \newline
\mbox{~~~~~~~~~~~~}\mbox{~~~~~~~~~~~~~~~~}+ P[R = 3] + P[R = 4] ... \newline
\mbox{~~~~~~~~~~~~}\mbox{~~~~~~~~~~~~~~~~~~~~~~~~~~~~~~~~}+ P[R = 4] ... \newline
\mbox{~~~~~~~~~~~~}\mbox{~~~~~~~~~~~~~~~~~~~~~~~~~~~~~~~~}+ P[R = 4] ... \newline
\mbox{~~~~~~~~~~~~}\mbox{~~~~~~~~~~~~~~~~~~~~~~~~~~~~~~~~}+ P[R = 4] ... \newline
\mbox{~~~~~~~~~~~~}\mbox{~~~~~~~~~~~~~~~~~~~~~~~~~~~~~~~~}+ P[R = 4] ... \newline
\mbox{~~~~~~~~~~~~}\mbox{~~~~~~~~~~~~~~~~~~~~~~~~~~~~~~~~}+ P[R = 4] ... \newline
= (1^2 - 0^2)P[R \geq 1] + (2^2 - 1^2)P[R \geq 2] + (3^2 - 2^2) P[R \geq 3] + (4^2 - 3^2) P[R \geq 4] ...
\newline
\newline$
This pattern and its continuation should be clear. Using this, we can now write $E[R^2]$ as a function
of $P[R > k]$, which we computed in part $1$:
\newline
\newline
$
E[R^2] =\newline
= \sum\limits_{k=1}^{n} P[R \geq k] (k^2 - (k-1)^2)  = \sum\limits_{k=1}^{n} P[R \geq k] (k^2 - k^2 + 2k - 1)
\newline
= \sum\limits_{k=1}^{n} P[R \geq k] (2k - 1) = \sum\limits_{k=1}^{n} P[R > k-1] (2k - 1)
\newline
= \sum\limits_{k=0}^{n-1} P[R > k] (2k + 1) \leq \sum\limits_{k=0}^{n} P[R > k] (2k + 1)
\newline
\leq \sum\limits_{k=0}^{n} P[R > k] (2k + 1)
\newline
\leq 2 \sum\limits_{k=0}^{n} k\cdot P[R > k] + \sum\limits_{k=0}^{n} P[R > k]$.
\newline
\newline
We will now compute both terms in the final expression above. Note that because $k$ goes from $0$ to $n$, we can exchange all appearances of
$k$ with $n-k$ without changing anything else:
\newline
\newline
$2 \sum\limits_{k=0}^{n} k\cdot P[R > k] 
\newline = 2\sum\limits_{k=0}^{n} \frac{n!k}{(n-k)! n^k} = 2 n!\sum\limits_{k=0}^{n} \frac{k}{(n-k)! n^k}
= 2 n!\sum\limits_{k=0}^{n} \frac{(n-k)}{k! n^{n-k}} = 2 \frac{n!}{n^n}\sum\limits_{k=0}^{n} \frac{(n-k)n^k}{k!}
\newline = 2 \frac{n!}{n^n}\sum\limits_{k=0}^{n} \frac{n \cdot n^k}{k!} - 2 \frac{n!}{n^n}\sum\limits_{k=0}^{n} \frac{k \cdot n^k}{k!} 
= 2 \frac{n! n}{n^n}\sum\limits_{k=0}^{n} \frac{n^k}{k!} - 2 \frac{n!}{n^n}\sum\limits_{k=1}^{n} \frac{k \cdot n^k}{k!} + 0
= 2 \frac{n! n}{n^n}\sum\limits_{k=0}^{n} \frac{n^k}{k!} - 2 \frac{n!}{n^n}\sum\limits_{k=1}^{n} \frac{n^k}{(k-1)!}
\newline = 2 \frac{n! n}{n^n}\sum\limits_{k=0}^{n} \frac{n^k}{k!} - 2 \frac{n!}{n^n}\sum\limits_{k=0}^{n-1} \frac{n^{k+1}}{k!}
= 2 \frac{n! n}{n^n}\sum\limits_{k=0}^{n} \frac{n^k}{k!} - 2 \frac{n!n}{n^n}\sum\limits_{k=0}^{n-1} \frac{n^{k}}{k!}
= 2 \frac{n! n}{n^n}\cdot(\sum\limits_{k=0}^{n} \frac{n^k}{k!} - \sum\limits_{k=0}^{n-1} \frac{n^{k}}{k!})
\newline = 2 \frac{n!n}{n^n} \cdot \frac{n^n}{n!} = 2n$.
\newline
\newline
Now for the second term:
\newline
$\sum\limits_{k=0}^{n} P[R > k]
\newline = \sum\limits_{k=0}^{n} \frac{n!}{(n-k)! n^k} = n! \sum\limits_{k=0}^{n} \frac{1}{k! n^{n-k}} = \frac{n!}{n^n} \sum\limits_{k=0}^{n} \frac{n^k}{k!}
\leq \frac{n! e^n}{n^n} = \frac{c n^n e^n \sqrt{2 \pi n}}{n^n e^n} = c \sqrt{n}$ where $c$ is some constant.
\newline
\newline
Note the use of the Taylor series expansion for $e^n$.
Now combining the second term with the first term, we can give an upper bound for $E[R^2]$:
\newline
$E[R^2] = 2 \sum\limits_{k=0}^{n} k\cdot P[R > k] + \sum\limits_{k=0}^{n} P[R > k] \leq 2n + c\sqrt{n} 
\newline\rightarrow E[R^2] = O(n)$.
\newline
\newline
Now for the lower bound. From the notes (page 273 of LL), we have a closed form upper bound on the probability that
no two people have the same birthday in a set of $k$ people. But this is the same as
the event that, after $k$ balls have been drawn out of $365$ total distinct balls, no two are the same.
Thus, because the problems are fundamentally the same, we can use this closed form lower bound on
$P[R > k]$. Take it for granted that:
\newline
\newline
$P[R > k] \geq c e^{-k(k-1)/2n}$.
\newline
\newline
Also note that:
\newline
\newline
$P[R^2 > k^2] = P[R > k]$ for positive $k$ since $R$ is always positive.
\newline
\newline
Using this with Markov's inequality, we have:
\newline
\newline
$P[R^2 > k^2] \leq E[R^2]/k^2 \rightarrow P[R > k] \leq E[R^2]/k^2$.
\newline
\newline
Essentially what this says is if we know $P[R > k]$, then we
have implicitly established a lower bound on $E[R^2]/k^2$. So, since we don't really care about constant factors,
let us pick a value for $P[R > k]$. Say $P[R > k] = c_1$ such that $0 < c_1 < 1$. Now, using the lower
bound established in the notes, we can put a lower bound on $k$ and plug into the equation above. Doing the math:
\newline
\newline
$P[R > k] = c_1 \rightarrow c_2 e^{-k(k-1)/2n} \leq c_1$ AND $c_1 \leq E[R^2]/k^2$.
\newline
\newline
And we can use the first equation to bound $k$ and plug into the second to bound $E[R^2]$.
\newline
\newline
$c_2 e^{-k(k-1)/2n} \leq c_1 \rightarrow e^{-k(k-1)/2n} \leq c_1/c_2 \rightarrow -k(k-1)/2n \leq \ln(c_1/c_2) 
\newline\rightarrow k(k-1)/2n \geq -\ln(c_1/c_2) \rightarrow k(k-1) \geq 2n\ln(c_2/c_1) \rightarrow k^2 - k \geq 2n\ln(c_2/c_1)
\rightarrow k^2 \geq 2n\ln(c2/c1)
\newline \rightarrow k \geq \sqrt{n} \sqrt{2\ln(c_2/c_1)}$.
\newline
\newline
And finally, using $P[R > k] = c_1$ and $k \geq \sqrt{n} \sqrt{2\ln(c_2/c_1)}$ in our original equation from
Markov's inequality, we get:
\newline
\newline
$P[R > k] = c_1 \leq E[R^2]/k^2 \leq E[R^2]/(2n\ln(c_2/c_1)) 
\newline\rightarrow E[R^2] \geq n \cdot 2c_1\ln(c_2/c_1)$ for some constants $0 < c_1 < 1$ and $c_2 > 0
\newline\rightarrow E[R^2] = \Omega(n)$.
\newline
\newline
Thus, having shown that $E[R^2] = \Omega(n)$ and $E[R^2] = O(n)$, we can conclude that $E[R^2] = \Theta(n)$
and that, therefore, from the definition of $\Theta$:
\newline
\newline
$c_1 n \leq E[R^2] \leq c_2 n$ for some constants $c_1$ and $c_2$ and $n$ sufficiently large,
\newline
\newline
therefore bounding $E[R^2]$ within constant factors.

%end sol 
\pagebreak{}
\thispagestyle{empty}

\begin{tabular*}{1\textwidth}{@{\extracolsep{\fill}}lr}
\textbf{ID:} 03 & \textbf{Collaborator \#1:} Last Name, First Name\tabularnewline
\textbf{Name:} Al-Naji, Nader & \textbf{Collaborator \#2:} Last Name, First Name\tabularnewline
\hline 
\end{tabular*}

\medskip{}


\begin{center}
\begin{Large}\textbf{Solution to HW 5, Problem 3}\end{Large}
\par\end{center}

\begin{center}
\begin{large}\textbf{COS 340 - Spring 2012}\end{large}
\par\end{center}


\bigskip{}


%begin sol
\textbf{1. Consider a set of positive integers $x_1, ..., x_l$ in $\{1, ..., n\}$ and define the random variable 
$X = b_1x_1 + ... + b_lx_l$, where the $b_i$'s are independent random variables equal to $0$ and $1$ with
probability $1/2$. Prove that, for any $\lambda > 1$,
\newline
\newline
$P[|X - E[X]| \geq \lambda n \sqrt{l}/2] \leq 1/\lambda^2$.}
\newline
\newline
Let $\sigma = \sqrt{Var[X]}$ and let $c= \lambda n \sqrt{l}/2 / \sigma$. If we choose this particular setting for $c$,
then we have $c\sigma = \lambda n \sqrt{l}/2$ and, further, by substituting $c\sigma$ in for $\lambda n \sqrt{l}/2$ and using
 the Chebyshev inequality, since $\lambda > 1$, it follows that:
\newline
\newline
$P[|X - E[X]| \geq \lambda n \sqrt{l}/2] = P[|X - E[X]| \geq c \sigma] \leq 1/c^2 = \frac{\sigma^2}{(1/4)\lambda^2 n^2 l} = (1/\lambda^2) \cdot \frac{\sigma^2}{(1/4)n^2 l}
\newline\rightarrow P[|X - E[X]| \geq \lambda n \sqrt{l}/2] \leq (1/\lambda^2) \cdot \frac{\sigma^2}{(1/4)n^2 l} \leq (1/\lambda^2)$ IF $\frac{\sigma^2}{(1/4)n^2 l} \leq 1$ ALWAYS.
\newline
\newline
So now all we have to do is show that $\frac{\sigma^2}{(1/4)n^2 l}$ is always less than or equal to $1$ and we will have our proof, since if this coefficient on
$(1/\lambda^2)$ is less than or equal to $1$, then the whole quantity $((1/\lambda^2) \frac{\sigma^2}{(1/4)n^2 l})$ will be less than or equal to $(1/\lambda^2)$ since $\lambda > 1$.
\newline
\newline
Now we show that $\sigma^2 = Var[X] \leq (1/4)n^2 l$. Note that since the $b_i$ are independent, the variance of their sum is equal to 
the sum of their variances.
\newline
\newline
$Var[X] = Var[\sum\limits_{i=1}^{l} b_i x_i] = \sum\limits_{i=1}^{l} Var[b_i x_i]$ because the $b_i$ are independent of each other
$\newline\rightarrow \sum\limits_{i=1}^{l} Var[b_i x_i] = \sum\limits_{i=1}^{l} x_i^2 Var[b_i] = \sum\limits_{i=1}^{l} x_i^2 Var[b_1]$ since the $b_i$ are identically distributed $\rightarrow Var[b_i] = Var[b_1]$ $\forall i$.
\newline
\newline
Now we compute $Var[b_1]$ to complete the calculation:
\newline
\newline
$Var[b_1] = E[b_1^2] - E[b_1]^2 = ((1/2)(1^2)) - ((1/2)(1))^2 = 1/2 - 1/4 = 1/4$. And we have:
\newline
\newline
$Var[X] = (1/4) \sum\limits_{i=1}^{l} x_i^2$.
\newline
\newline
Finally, because we are told in the beginning that the maximum value each of the $x_i$ can take on is $n$, we can compute the upper bound of the 
variance by assuming that all of the $x_i$ take on their maximum value and get:
\newline
\newline
$Var[X] = (1/4) \sum\limits_{i=1}^{l} x_i^2 \leq (1/4) \sum\limits_{i=1}^{l} n^2 = (1/4) n^2 l
\newline\rightarrow \sigma^2 \leq (1/4)n^2l$.
\newline
\newline
This is exactly the result we were looking for and it lets us complete the proof. To resummarize:
\newline
\newline
$P[|X - E[X]| \geq \lambda n \sqrt{l}/2] = P[|X - E[X]| \geq c \sigma] \leq 1/c^2 = \frac{\sigma^2}{(1/4)\lambda^2 n^2 l} = (1/\lambda^2) \cdot \frac{\sigma^2}{(1/4)n^2 l}
\newline\rightarrow P[|X - E[X]| \geq \lambda n \sqrt{l}/2] \leq (1/\lambda^2) \cdot \frac{\sigma^2}{(1/4)n^2 l} \leq (1/\lambda^2)$ since $\frac{\sigma^2}{(1/4)n^2 l} \leq 1$ and $\lambda > 1
\newline
\newline\rightarrow P[|X - E[X]| \geq \lambda n \sqrt{l}/2] \leq (1/\lambda^2)$.
\newline
\newline
\textbf{2. A set of positive integers $x_1, ..., x_l$ is said to have distinct sums if all the numbers $\sum\limits_{i \in S}^{} x_i$
are distinct, for all nonempty subsets $S \subset \{1, ..., l\}$. Let $f(n)$ be the maximal value of $l$ such that there exists
a set of $l$ positive integers in $\{1, ..., n\}$ with distinct sums. Prove that
\newline
\newline
$f(n) < 2 \lg n$.  }
\newline
\newline
First, note that because $l$ is a set, it cannot contain more than one instance of each number from $1$ to $n$. Thus, the size of 
$l$ cannot be exceed $n$ and, further, at its maximum size, $l$ will contain one of each of $\{1, ..., n\}$. So the maximum sum 
that any subset of $\{1, ..., n\}$ can achieve is the sum of all numbers between $1$ and $n$ and thus the sum of any subset of 
$\{1, ..., n\}$ must take on a value between $[1, \sum\limits_{i=1}^{n} i] = [1, (1/2)n(n+1)]$. 
\newline
But the number of subsets of a set of size $l$ is equal to $2^l$. So if $l$ is too large, namely if $2^l > (1/2)n(n+1)$, then, by the pigeonhole principle, there
will be some pair of subsets of $l$ that have the same sum. Thus, for a set of size $l$ constructed from the numbers $\{1, ..., n\}$ to have distinct sums, it must satisfy
the following. Note that throughout the proof, the fact that all of our quantities are positive is used implicitly in inequalities and such:
\newline
\newline
$2^l \leq (1/2)n(n+1) \rightarrow l \leq \lg 1/2 + \lg n + \lg (n+1) \rightarrow l \leq \lg n + (\lg (n+1) - 1)$.
\newline
\newline
Now we just need to show that $(\lg (n+1) - 1) < \lg n$ and we have our proof:
\newline
\newline
$\lg (n+1) - 1 < \lg n \rightarrow \lg(n+1) - \lg n < 1 \rightarrow \lg((n+1)/n) < 1 \rightarrow (n+1)/n < 2 \rightarrow n+1 < 2n 
\newline\rightarrow n < 2n - 1 \rightarrow n < n(2 - 1/n) \leftarrow$ true if
$ (2 - 1/n) > 1$ since $n$ is positive and coefficient on $n$ being greater than $1$ will make the quantity $n(2 - 1/n)$ greater than $n$.
So showing it:
$\newline\rightarrow(2 - 1/n) > 1\rightarrow 2 > 1 + 1/n \rightarrow 1 > 1/n 
\newline\rightarrow n > 1$. 
\newline\newline
So as long as $n$ is greater than $1$, a safe assumption, the inequality $\lg (n+1) - 1 < \lg n$ holds and we can finish our proof:
\newline
\newline
$l \leq \lg n + (\lg (n+1) - 1) < \lg n + \lg n
\newline\rightarrow l < 2 \lg n$.
%end sol

\pagebreak{}

\thispagestyle{empty}

\begin{tabular*}{1\textwidth}{@{\extracolsep{\fill}}lr}
\textbf{ID:} 03 & \textbf{Collaborator \#1:} Last Name, First Name\tabularnewline
\textbf{Name:} Al-Naji, Nader & \textbf{Collaborator \#2:} Last Name, First Name\tabularnewline
\hline 
\end{tabular*}

\medskip{}


\begin{center}
\begin{Large}\textbf{Solution to HW 5, Problem 4}\end{Large}
\par\end{center}

\begin{center}
\begin{large}\textbf{COS 340 - Spring 2012}\end{large}
\par\end{center}

\bigskip{}

%begin sol
TilghmanPCS has introduced a new pay-as-you-go data plan for Princeton: Customers can buy different kinds of
bundles every month (and several in a month), where each bundle has a fixed price and buys them the ability to 
transfer a certain number of megabytes in the month that the bundle is bought. Unused transfer capacity expires at the end of the month. Tilghman offers a choice of bundles labelled type 1, type 2, ..., type $i$ for every integer $i > 0$. A type $i$ bundle costs $i$ dollars upfront and allows the customer to transfer $i^2$ megabytes in a month. Bundles can be bought at any time in the month (and the same bundle can be bought repeatedly if required) - additional bundles add to the existing capacity already purchased. However, unused megabytes expire at the end of the month. Your goal is to design an online algorithm for the customer to determine which bundles to purchase based on the current usage. Assume that whenever the current transfer capacity is exhausted, the customer has to decide which bundle to purchase for additional data transfers. 
\newline
\newline
\newline
\textbf{1. Determine the optimal offline solution for this problem. }
\newline
\newline
The optimal offline algorithm knows, a priori, exactly how many megabytes a user will require during the month. Note that we hereafter refer to packages as the individual type $i$ items a user can purchase and refer to bundles as being made up of one or more packages. Let the number of megabytes the
user requires in a particular month be denoted by $n$ and let the amount of money the offline algorithm decides to spend be denoted
by $k$. Let $V_{k}$ be the set of all possible numbers of megabytes that can be purchased with $k$ dollars. So, $V_1 = \{1\}$, for example,
since $1$ dollar can only purchase exactly $1$ megabyte. $V_2 = \{2, 4\}$, on the other hand, since a bundle costing
$2$ dollars that consists of $2$ type $1$ packages gets the user $2$ megabytes, and a bundle costing $2$ dollars consisting of $1$ type $2$ package 
gets the user $4$ megabytes. With this notation down, it is the job of the optimal offline algorithm to find a bundle $b$ of cost $k$ such that:
\newline
\newline
$\sup(V_{k-1}) < n$ and (number of megabytes purchased by bundle $b$)$ \geq n$.
\newline
\newline
In words, the number of megabytes purchased by bundle $b$ must be at least $n$ and, if bundle $b$ costs $k$ dollars, there must not exist a bundle 
costing $k-1$ dollars that can also purchase $n$ megabytes. 
\newline
\newline
The most cost-effective way to purchase at least $n$ megabytes is to buy a type $\lceil \sqrt{n} \rceil$ bundle. We will now show that this
is the case by proving that no bundle costing $k$ dollars of consisting of more than one package can purchase more megabytes than a single type $k$ 
package.
\newline
\newline
Consider a bundle costing $k$ dollars consisting of a single package and another bundle consisting of two packages costing $i$ dollars and $j$ dollars respectively such that
$i + j = k$. We want to show that the number of megabytes purchased by the first bundle is more than the number of megabytes purchased by the second
bundle for all choices of $i, j | i+j = k$. So, we show:
\newline
\newline
$k^2 > i^2 + j^2$ $ \forall i, j $ $ | $ $ i + j = k 
\newline\rightarrow k^2 > i^2 + (k - i)^2
\newline\rightarrow k^2 > i^2 + k^2 - 2k i + i^2
\newline\rightarrow k^2 > k^2 - 2(ki - i^2)
\newline$Note that $ki > i^2$ since we defined $i$ to be strictly less than $k$ and implies $ \rightarrow (ki - i^2) > 0$ $
\newline\rightarrow k^2 > k^2 - 2($some positive value$)$; a true statement, which implies$
\newline\newline\rightarrow k^2 > i^2 + j^2$ $ \forall i, j $ $ | $ $ i + j = k$ is true.
\newline
\newline
Having shown this inequality, we now interpret its result. This inequality shows that any bundle costing $k$ dollars that consists of more than one
package will purchase strictly less than a bundle consisting of a single package that costs $k$ dollars. This result shows that breaking a single package up into multiple packages
lowers the buying power of that package in general. So a bundle costing $k$ dollars consisting of a single package will purchase more megabytes than a bundle 
consisting of $i > 1$ packages in general since breaking up the single package causes its buying power and, consequently, the buying power of the whole bundle, to decrease.
\newline
\newline
This result serves to show that the most cost-effective bundle that can purchase $n$ megabytes when $n$ is a perfect square is a type $\sqrt{n}$ bundle since every other
bundle consisting of more than a single package will purchase strictly fewer megabytes than this bundle, and therefore not satisfy the user.
When $n$ is not a perfect square, we know that $n > (\lfloor \sqrt{n} \rfloor)^2$ so a bundle costing $\lfloor \sqrt{n} \rfloor$ cannot purchase enough megabytes
to satisfy the user's need. Further, using our result from above, because this single-package bundle costing $\lfloor \sqrt{n} \rfloor$ dollars cannot purchase $n$ megabytes, we know that NO bundle costing $\lfloor \sqrt{n} \rfloor$ dollars can purchase $n$ megabytes and so the best we can do is to buy a bundle costing $\lceil \sqrt{n} \rceil = \lfloor \sqrt{n} \rfloor + 1$ dollars since $n$ is not a perfect square. While it isn't always strictly necessary that this bundle be a single-package type $\lceil \sqrt{n} \rceil$ bundle, we may as well purchase this particular bundle in the case that $n$ is not a perfect square because we don't care about wasted bandwidth. For example, if $n = 82$, we could buy a type $10$ bundle, or a type $9$ and type $1$ bundle, but we choose the type $10$ bundle in this case.
\newline
\newline
To summarize the optimal offline algorithm:
\newline
Always purchase a single type $\lceil \sqrt{n} \rceil$ package at the beginning of the month for a cost of $\lceil \sqrt{n} \rceil$ dollars. 
\newline
\newline
\textbf{2. Design and analyze an online algorithm with constant competitive ration.}
\newline
\newline
We will analyze the following algorithm:
\newline
\newline
1. Start out each month with $i = 0$ and purchase a single type $2^{i}$ package.
\newline
2. If/when this this package gets used up, increment $i$ (so $i = i + 1$) and purchase a type $2^i$ package.
\newline
\newline
So the first package purchased will be of type $1$. If the user exhausts this package, they will purchase a type $2$ package, then if that gets used up a type $4$ package, and so on.
In order to compute this package's cost as a function of $n$, we note two facts. In the following, $l$ is the number of times a user exhausts a package:
\newline
$k = \sum\limits_{i=0}^{l} 2^i$ and $n \geq \sum\limits_{i=0}^{l-1} 2^{2i}$.
\newline
The first fact is true because if we run out $l$ times, it means we bought $l + 1$ packages each costing $2^i$ for $i$ from $0$ to $l$.
\newline
The second fact is true because in order for us to have purchased our last package, we will have to have run out when using our second to last
package and so $n$ must be greater than the sum of our package values if we don't include the value of the last package we bought.
\newline
\newline
We will now solve for $l$ in the second inequality and plug it into the first equality in order to get $k$ as a function of $l$.
\newline
\newline
$n \geq \sum\limits_{i=0}^{l-1} 2^{2i}
\newline\rightarrow n \geq (1 - 4^l)/(1-4)
\newline\rightarrow n \geq (4^l - 1)/(3)
\newline\rightarrow 3n \geq 4^l - 1 \rightarrow 3n \geq 4^l
\newline\rightarrow \lg(3n) \geq l \cdot \lg 4 \rightarrow \lg(3n) \geq l \cdot 2
\newline\newline\rightarrow l \leq \frac{\lg(3n)}{2}$
\newline
\newline
Plugging this into the first equation, we have:
\newline
\newline
$k = \sum\limits_{i=0}^{l} 2^i = (1 - 2^{l+1})/(1 - 2) = (2^{l+1} - 1)/(1)
\newline\rightarrow k = 2\cdot 2^{l} - 1 \leq 2 \cdot 2^{\frac{\lg(3n)}{2}} - 1 = 2 \cdot (2^{\lg(3n)})^{1/2} - 1 = 2 \cdot \sqrt{3n} - 1
\newline
\newline\rightarrow k \leq 2 \sqrt{3n}$
\newline
\newline
Finally having bounded $k$ for our online algorithm, we can compare the performance of the online algorithm to the offline algorithm. Note that it
doesn't matter how quickly or how slowly the user uses up megabytes, only how many, denoted by $n$. So we have:
\newline
\newline
$C_{opt}[n] = \lceil \sqrt{n} \rceil \geq \sqrt{n}$ and \newline
$C_A[n] \leq 2 \sqrt{3n}$. So,
\newline
$C_A/C_{opt} \leq \frac{2   \sqrt{3n}}{\sqrt{n}} = 2  \sqrt{3}$.
\newline
\newline
So this algorithm is $(2 \sqrt{3})$-competitive.
\newline
%end sol
\pagebreak{}

\end{document}
