%% LyX 2.0.3 created this file.  For more info, see http://www.lyx.org/.
%% Do not edit unless you really know what you are doing.
\documentclass[english]{article}
\usepackage[T1]{fontenc}
\usepackage[latin9]{inputenc}
\usepackage{geometry}
\geometry{verbose,tmargin=0.75in,bmargin=0.75in,lmargin=0.75in,rmargin=1in}

\makeatletter

%%%%%%%%%%%%%%%%%%%%%%%%%%%%%% LyX specific LaTeX commands.
%% Because html converters don't know tabularnewline
\providecommand{\tabularnewline}{\\}

\makeatother

\usepackage{babel}
\begin{document}
\thispagestyle{empty}

\begin{tabular*}{1\textwidth}{@{\extracolsep{\fill}}lr}
\textbf{ID:} 03 & \textbf{Collaborator \#1:} Last Name, First Name\tabularnewline
\textbf{Name:} Al-Naji, Nader & \textbf{Collaborator \#2:} Last Name, First Name\tabularnewline
\hline 
\end{tabular*}

\medskip{}


\begin{center}
\begin{Large}\textbf{Solution to TA1, Problem 1}\end{Large}
\par\end{center}

\begin{center}
\begin{large}\textbf{COS 445 - Fall 2012}\end{large}
\par\end{center}

\bigskip{}


% Begin sol%
A degree $k$ polynomial with integer coefficients is an expression of the form $\sum\limits_{i = 0}^{k}c_i\cdot x^i$ where coefficients $c_i$ are integers and $c_k \neq 0$.
\newline
\newline
1. Prove that the set of all such polynomials where the degree $k$ is fixed, is countable. 
\newline
\newline
The hint suggests we use induction. We start proving that a degree one polynomial is countable and then go from there.
\newline
\newline
Let $Q(k) = $ the set of all polynomials of degree $k$ is countable. 
\newline
\newline
\textbf{Base: } $k = 1$
\newline
\newline
In the case where $k = 1$, we have a polynomial of the form: $c_1\cdot x + c_0$ where $c_1$ and $c_0$ are both integers. We can count all such polynomials in the same way we counted the rational numbers. We consider all positive values of $c_1$ and all positive values of $c_0$, put them on perpendicular axes and snake around. This assures that we will eventually cover all positive pairs of $c_1$ and $c_0$. To get to the negative pairs, we simply add the negative values in between the positive values and put zero at the front without loss of generality. The following is an illustration of how we count these.
% End sol%

\pagebreak{}

\thispagestyle{empty}

\begin{tabular*}{1\textwidth}{@{\extracolsep{\fill}}lr}
\textbf{ID:} 03 & \textbf{Collaborator \#1:} Last Name, First Name\tabularnewline
\textbf{Name:} Al-Naji, Nader & \textbf{Collaborator \#2:} Last Name, First Name\tabularnewline
\hline 
\end{tabular*}

\medskip{}


\begin{center}
\begin{Large}\textbf{Solution to TA1, Problem 2}\end{Large}
\par\end{center}

\begin{center}
\begin{large}\textbf{COS 445 - Fall 2012}\end{large}
\par\end{center}

\bigskip{}


% Begin sol%

% End sol%

\pagebreak{}
\thispagestyle{empty}

\begin{tabular*}{1\textwidth}{@{\extracolsep{\fill}}lr}
\textbf{ID:} 03 & \textbf{Collaborator \#1:} Last Name, First Name\tabularnewline
\textbf{Name:} Al-Naji, Nader & \textbf{Collaborator \#2:} Last Name, First Name\tabularnewline
\hline 
\end{tabular*}

\medskip{}


\begin{center}
\begin{Large}\textbf{Solution to TA1, Problem 3}\end{Large}
\par\end{center}

\begin{center}
\begin{large}\textbf{COS 445 - Fall 2012}\end{large}
\par\end{center}

\bigskip{}


% Begin sol%

% End sol%

\pagebreak{}

\thispagestyle{empty}

\begin{tabular*}{1\textwidth}{@{\extracolsep{\fill}}lr}
\textbf{ID:} 03 & \textbf{Collaborator \#1:} Last Name, First Name\tabularnewline
\textbf{Name:} Al-Naji, Nader & \textbf{Collaborator \#2:} Last Name, First Name\tabularnewline
\hline 
\end{tabular*}

\medskip{}


\begin{center}
\begin{Large}\textbf{Solution to TA1, Problem 4}\end{Large}
\par\end{center}

\begin{center}
\begin{large}\textbf{COS 445 - Fall 2012}\end{large}
\par\end{center}

\bigskip{}


% Begin sol%

% End sol%

\pagebreak{}

\end{document}
