% Use this template to write your solutions

\documentclass[12pt]{article}
\usepackage{slashbox}
% Set the margins
%
\setlength{\textheight}{8.5in}
\setlength{\headheight}{.25in}
\setlength{\headsep}{.25in}
\setlength{\topmargin}{0in}
\setlength{\textwidth}{6.5in}
\setlength{\oddsidemargin}{0in}
\setlength{\evensidemargin}{0in}

% Macros
\newcommand{\myN}{\hbox{N\hspace*{-.9em}I\hspace*{.4em}}}
\newcommand{\myZ}{\hbox{Z}^+}
\newcommand{\myR}{\hbox{R}}

\newcommand{\myfunction}[3]
{${#1} : {#2} \rightarrow {#3}$ }

\newcommand{\myzrfunction}[1]
{\myfunction{#1}{{\myZ}}{{\myR}}}


% Formating Macros

\newcommand{\myheader}[4]
{\vspace*{-0.5in}
\noindent
{#1} \hfill {#3}

\noindent
{#2} \hfill {#4}

\noindent
\rule[8pt]{\textwidth}{1pt}

\vspace{1ex} 
}  % end \myheader 

\newcommand{\myalgsheader}[0]
{\myheader
{ {\bf{COS 340}} }
{ {\bf{Spring 2012}} }
{ {\bf{Collaborator 1}} : Diao, Lawrence }
}

% Running head (goes at top of each page, beginning with page 2.
% Must precede by \pagestyle{myheadings}.
\newcommand{\myrunninghead}[2]
{\markright{{\it {#1}, {#2}}}}

\newcommand{\myrunningalgshead}[2]
{\myrunninghead{COS 340 }{{#1}}}

\newcommand{\myrunninghwhead}[2]
{\myrunningalgshead{Solution to HW {#1}, Problem {#2}}}

\newcommand{\mytitle}[1]
{\begin{center}
{\large {\bf {#1}}}
\end{center}}

\newcommand{\myhwtitle}[3]
{\begin{center}
{\large {\bf Solution to HW {#1}, Problem {#2}}}\\
\medskip 
{\it {#3}} % Name goes here
\end{center}}

\newcommand{\mysection}[1]
{\noindent {\bf {#1}}}

%%%%%% Begin document with header and title %%%%%%%%%%%%%%%%%%%%%%%%%

\begin{document}

\myalgsheader

\pagestyle{plain}

\myhwtitle{1}{1}{Al-Naji, Nader}
% Example : \myhwtitle{1}{4}{Your name here}

\bigskip

% begin Solution 
Let $S = \{1,2, ...,n\}$. How many ordered pairs $(A,B)$ of subsets of S are there that satisfy $A \subseteq B$?
\newline
\newline
Let $B_k$ consist of the subsets of $S$ that contain exactly $k$ elements. If we fix $k$, then $|B_k| = $$n \choose k$ since there are
$n \choose k$ different ways to choose $k$ elements from a set of $n$. Now, each of these $k$ element subsets of $S$ has $2^k$ subsets of its
own, a fact we showed in class. So if we wish to enumerate all tuples $(A_{j}, B_{k,i})$ such that $A_{j} \subseteq B_{k,i}$ for some set $A_j$ with $B_{k,i} \in B_k$, then
there will be $2^k \times $$n \choose k$ of them since there are $n \choose k$ elements in the set $B_k$ and each one has
$2^k$ subsets. We get the quantity we want by summing over $k$. $\sum_{k=0}^n 2^n \times $$n \choose k$$ = $$\sum_{k=0}^n 2^n \times 1^{n-k} $$n \choose k$
$ = $$3^n$ tuples of $(A,B)$ such that $A, B \subseteq S$ and $A \subseteq B$.
\newline
\newline
Another way to think about this is to consider the the set of all strings of length $n$ having the numbers $0$, $1$, or $2$ as digits. By the 
generalized product rule, there are $3 \time 3 \times ... \times 3$ = $3^n$ of these strings. Now, if we think of each digit in one of these
strings as representing an element of the set $S$ and having it take on the digit $0$ as meaning the element is in set $S$ but not in
sets $A or B$, the digit $1$ as representing the element being in $B$ but not in $A$, and the digit $2$ as representing the element being
in both $A and B$, then we see that these strings have a one-to-one mapping to tuples $(A, B)$ from above and, by the bijection rule,
that there are $3^n$ of these tuples.

\newpage
\myhwtitle{1}{2}{Al-Naji, Nader}
% Example : \myhwtitle{1}{4}{Your name here}

\bigskip

Prove that given $n > 0$ and a set of $n+1$ positive integers, none of them exceeding $2n$, there is at least one integer in this set that divides 
another integer in this set.
\newline
\newline
First, note that all even integers can be written in the form $k \times 2^b$ for some b where k is an odd number smaller than the integer itself and $b >= 0$. To see this, note that a factor that is even can always be made odd by pulling out a two. So consider all pairs of the form $(k, ak)$ where $k$ is an odd number and $a$ has the form $2^b$ for some $b \geq 0$. In a set of 2n numbers with $n > 0$, there are
n different values that k can take on, one for each of the odd numbers in the set of numbers $[0,2n]$. Further, note that every number in the set $[0,2n]$ belongs to exactly one of
these n different groups: a number is either odd, in which case it has its own group where it is the leftmost part of the tuple, or a number is even, in which case it can be broken down
into a multiple of $2^b$ and a distinct odd number $k$ and put into the $k^{th}$ group as the rightmost part of the tuple. Finally, note that if two numbers are in the same
group, then one of them divides the other. Since there are only $n$ different groups in a set of $2n$ numbers, then by the pigeonhole principle, selecting a set of $n+1$ numbers from
this set results in at least one collision with two of the numbers being in the same group and, therefore, one number dividing another. 

\newpage
\myhwtitle{1}{3}{Al-Naji, Nader}
% Example : \myhwtitle{1}{4}{Your name here}

\bigskip

A chocolate bar consists of $n \times m$ (where $n,m \geq 1$) square pieces arranged in a rectangle. The square in the upper left corner is poisoned. Two players (Alice and Bob)
are taking turns breaking the bar into two rectangular bars along the lines, and eating the non-poisoned part. The player left with the lone poisoned piece loses the game.
Alice makes the first move. 
\newline
\newline
(a) For each pair $1 \leq n,m \leq 5$ which player has the winning strategy?
\newline
\newline
\begin{tabular}{r|ccccc}
\backslashbox[10pt][l]{n}{m} & 1 & 2 & 3 & 4 & 5\\\hline
1 & Bob & Alice & Alice & Alice & Alice\\
2 & Alice & Bob & Alice & Alice & Alice\\
3 & Alice & Alice & Bob & Alice & Alice\\
4 &	Alice	&	Alice  & Alice		& Bob & Alice\\
5 &	Alice	&	Alice  & Alice		& Alice& Bob
\end{tabular} 
\newline
\newline
The $1 \times 1$ case is trivial: Alice loses automatically. With any $n \times 1$ bar, Alice can turn the bar into a $1 \times 1$ and force Bob to lose so she 
wins the rest. If the bar is a $2 \times 2$, then all of Alice's move result in her losing since Bob can immediately turn the bar into a $1 \times 1$. With all the other $n \times 2$ moves, Alice can turn the bar into 
a square and force Bob to lose. Alice wins the $3 \times 1$ and $3 \times 2$ case by symmetry. With the $3 \times 3$ case, again all of her moves result in an 
automatic loss since Bob can immediately turn it into a $2 \times 2$ or $1 \times 1$ and force a loss. With the rest of the $n \times 3$, Alice can turn the bar
into a $3 \times 3$ for Bob and force him to lose. The first $3$ of the $n \times 4$ case come by symmetry again and $4 \times 4$ results in a loss since Bob can
immediately turn the bar into a $k \times k$ and force Alice to lose. Alice wins the $4 \times 5$ bar by turning it into a $4 \times 4$ for Bob. Finally, the
first $4$ of the $n \times 5$ case come by symmetry. In the $5 \times 5$ case Alice loses because Bob can turn the bar into a square after her turn, which makes
her lose.
\newline
\newline
(b) Who has the winning strategy for arbitrary $n,m \geq 1$? Prove your answer.
\newline
\newline
If $n = m$, Bob; otherwise Alice.
\newline
\newline
Proof: 
\newline
First note that a move in the game serves only to decrement a single dimension $n $ or $m$ of the bar, but never both dimensions simultaneously, since
the lines are horizontal and vertical and breaking one of these lines has no effect on the bar's size along the line perpendicular to it. Now,
let:
\newline
\newline
$P(n) = $ ``if the bar is $n \times m$ with $n = m$ and it is player one's turn, player two wins. Otherwise, if $n < m$ and it is player two's turn, player two wins.''
\newline
Note that we use $n < m$ in the second half instead of $n \neq m$ because otherwise $P(n)$ where $n = a$ and $m = b$, and $P(m)$ where $n = b$ and $m = a$ , would be saying the same thing whereas here they are two distinct cases. Also note that this statement
is symmetric where player one is the person whose turn it is to make a move, i.e. player one could be Alice or Bob depending on whose turn it is.
\newline
\newline
\textbf{Base}: $P(n) = 1$
\newline
\newline
If $n = m = 1$ and it is Alice's turn, then Bob wins since Alice has to eat the poison piece automatically. 
\newline
\newline
Otherwise, if $n \neq m$, then 
$m > 1$, since $m$ can't be less than 1, and Alice could decrement $m$ on her move to $1$ and force an automatic loss for Bob.
\newline
\newline
\textbf{Step}: assume $P(1), ... , P(n)$ is true. Need to show $P(n+1)$ is true.
\newline
\newline
For the case of an $m_1 \times n+1$ bar, where $m_1 = n+1$ and it is Alice's turn, Alice has to make a move and, therefore, decrement either dimension. By symmetry, it
doesn't matter which dimension she chooses to decrement, but when she does make her move she will generate an  $m_2 \times n+1$ square with $m_2 < n+1$ and $m_2 \in [0,n]$ and
hand this square over to Bob. Now Bob will be given an $m_2 \times n+1$ where $m_2 < n+1$ and $m_2 \in [0,n]$ and will win by $P(m_2)$ which we assume is true.
\newline
\newline
For the case of an $m_1 \times n+1$ bar, where $m_1 > n+1$ and it is Alice's turn, Alice can choose to decrement $m_1$ down to $n+1$ and hand over an $(n+1) \times (n+1)$ sized square over to 
Bob. Bob will then lose by the statement above for $m_1 = n+1$.

\newpage
\myhwtitle{1}{4}{Al-Naji, Nader}
% Example : \myhwtitle{1}{4}{Your name here}

\bigskip

Arrange the following functions in increasing order of growth rate, so that for two consecutive functions $f(n), g(n)$ in your sequence, either $f(n) = o(g(n)) or f(n) = Theta(g(n)).$
Explain the relationships of all the adjacent pairs of functions in your order. Assume here that log is a logarithm to base 2 and ln is a logarithm to base $e$.
\begin{center}
\Large $n!$  $\log n$  $\log n^2$  $n^{10}$  $\ln n$  $n \choose 10$  $2^{\log n}$  $2^{\ln^2 n}$  $(\log n)^n$  $n^{(\log n)}$
\end{center}
The following is a proper ordering as will be shown:
\begin{center}
\Large $\ln n$ $\log n$ $\log n^2$ $2^{\log n}$ $n \choose 10$ $n^{10}$ $2^{\ln^2 n}$ $n^{(\log n)}$ $(\log n)^n$ $n!$
\end{center}
\section{$\log n$ = $\Theta(\ln n)$}
$\lim_{n\to\infty} \frac{\ln n}{\log n} = \lim_{n\to\infty} \ln n \times \frac{\ln 2}{\ln n} = \lim_{n\to\infty} \ln 2 = \ln 2$
\newline
Because $\lim_{n\to\infty} \frac{\ln n}{\log n} \neq 0 \Rightarrow \ln n \neq o(\log n)$
\newline
and $\lim_{n\to\infty} \frac{\ln n}{\log n} \neq \infty \Rightarrow \ln n \neq \omega(\log n)$,
\newline
$\log n$ = $\Theta(\ln n)$
\section{$\log n^2$ = $\Theta(\log n)$}
$\lim_{n\to\infty} \frac{\log n}{\log n^2} = \lim_{n\to\infty} \frac{\log n}{2 \times \log n} = \frac{1}{2}$
\newline
Because $\lim_{n\to\infty} \frac{\log n}{\log n^2} \neq 0 \Rightarrow \log n \neq o(\log n^2)$
\newline
and $\lim_{n\to\infty} \frac{\log n}{\log n^2} \neq \infty \Rightarrow \log n \neq \omega(\log n^2)$,
\newline
$\log n$ = $\Theta(\ln n)$
\section{$\log n^2$ = $o(2^{\log n})$}
$\lim_{n\to\infty} \frac{\log n^2}{2^{\log n}} = 2 \times \lim_{n\to\infty} \frac{\log n}{n} = \frac{2}{\ln 2} \lim_{n\to\infty} \frac{\ln n}{n}$
\newline
We will use L'Hopital's rule. First, note that $\lim_{n\to\infty} \log n = \lim_{n\to\infty} \log n  = \infty$
and that $\frac{d}{dn} n = 1 \neq 0$ so L'Hopital's rule will hold. Applying it we get:
\newline
$\lim_{n\to\infty} \frac{\frac{d}{dn}\ln n}{\frac{d}{dn}n} = \lim_{n\to\infty} \frac{1}{n} = 0$ and $\lim_{n\to\infty} \frac{\log n^2}{2^{\log n}} = 0$
\newline
Thus, by the definition of $f(n) = o(g(n))$, $\log n^2$ = $o(2^{\log n})$.
\newline
\section{$2^{\log n}$ = $o($$n \choose {10}$$)$}
$\lim_{n\to\infty} \frac{2^{\log n}}{{n \choose {10}}} = \lim_{n\to\infty} \frac{n (n-10)! 10!}{n!}$
\newline
$ = \lim_{n\to\infty} \frac{n \times 10! \times (n-10)(n-11)...}{ (n)(n-1)(n-2)(n-3)(n-4)(n-5)(n-6)(n-7)(n-8)(n-9)(n-10)(n-11)...}$
\newline
$ = \lim_{n\to\infty} \frac{n \times 10!}{(n)(n-1)(n-2)(n-3)(n-4)(n-5)(n-6)(n-7)(n-8)(n-9)}$
\newline
$= \lim_{n\to\infty} \frac{10!}{(n-1)(n-2)(n-3)(n-4)(n-5)(n-6)(n-7)(n-8)(n-9)} = 0$
\newline
Thus, by the definition of $f(n) = o(g(n))$, $2^{\log n}$ = $o($$n \choose {10}$$)$.
\section{$n \choose {10}$ = $\Theta(n^{10})$}
$\lim_{n\to\infty} \frac{n^{10}}{{n \choose {10}}} = \lim_{n\to\infty} \frac{n^{10} (n-10)! 10!}{n!}$
\newline
$ = \lim_{n\to\infty} \frac{n^{10} \times 10! \times (n-10)(n-11)...}{ (n)(n-1)(n-2)(n-3)(n-4)(n-5)(n-6)(n-7)(n-8)(n-9)(n-10)(n-11)...}$
\newline
$ = \lim_{n\to\infty} \frac{n^{10} \times 10!}{(n)(n-1)(n-2)(n-3)(n-4)(n-5)(n-6)(n-7)(n-8)(n-9)}$
\newline
$= \lim_{n\to\infty} \frac{n^{10} \times 10!}{-362880 n + 1026576 n^2 - 1172700 n^3 + 723680 n^4 - 269325 n^5 + 
 63273 n^6 - 9450 n^7 + 870 n^8 - 45 n^9 + n^{10}}$
\newline
$= \lim_{n\to\infty} \frac{3628800 \times n^{10}}{n^{10} + ... - 362880 n} = 3628800$
\newline
\newline
Because $\lim_{n\to\infty} \frac{n^{10}}{{n \choose {10}}} \neq 0 \Rightarrow n^{10} \neq $ $o($$n \choose {10}$$)$
\newline
and $\lim_{n\to\infty} \frac{n^{10}}{{n \choose {10}}} \neq \infty \Rightarrow n^{10} \neq $ $\omega($$n \choose {10}$$)$,
\newline
${n \choose {10}}$ = $\Theta(n^{10})$
\section{$n^{10}$ = $o(2^{\ln^2 n})$}
First not that: $2^{\ln^2 n} = 2^{(\ln n)(\ln n)} = 2^{(\frac{\log n}{\log e})(\ln n)} = 2^{(\log n)(\frac{\ln n}{\log e})} = n^{\frac{\ln n}{\log e}}$
\newline
$\lim_{n\to\infty} \frac{n^{10}}{n^{\frac{\ln n}{\log e}}} = \lim_{n\to\infty} n^{10 - \frac{\ln n}{\log e}} = 0$
\newline
Thus, by the definition of $f(n) = o(g(n))$, $n^{10}$ = $o(2^{\ln^2 n})$
\section{$2^{\ln^2 n} = o(n^{\log n})$}
$\lim_{n\to\infty} \frac{n^{\frac{\ln n}{\log e}}}{n^{\log n}} = \lim_{n\to\infty} n^{\frac{\ln n}{\log e} - \log n} = \lim_{n\to\infty} n^{\frac{\ln n}{\log e} - \frac{\ln n}{\ln 2}}$
$= \lim_{n\to\infty} n^{\ln n \times (\frac{1}{\log e} - \frac{1}{\ln 2})}$
\newline
Note that: $\frac{1}{\log e} < \frac{1}{\ln 2}$  so...
\newline
$\lim_{n\to\infty} n^{\ln n \times (\frac{1}{\log e} - \frac{1}{\ln 2})} = 0$
\newline
Thus, by the definition of $f(n) = o(g(n))$, $2^{\ln^2 n} = o(n^{\log n})$
\section{$n^{\log n} = o((\log n)^n)$}
The root test states that a series with positive terms $\sum a_n$ will converge, which is the same as saying $\lim_{n\to\infty} a_n = 0$,
if $\sqrt[n]{a_n} < 1$. We will use this test here to show that $\lim_{n\to\infty} \frac{n^{\log n}}{(\log n)^n} = 0$.
\newline
\newline
$\lim_{n\to\infty} \sqrt[n]{\frac{n^{\log n}}{(\log n)^n}} = \lim_{n\to\infty} \frac{n^{\frac{\log n}{n}}}{(\log n)} = \lim_{n\to\infty} \frac{1}{\log n} = 0 < 1$
\newline
$\Rightarrow \lim_{n\to\infty} \frac{n^{\log n}}{(\log n)^n} = 0$
\newline
\newline
Thus, by the definition of $f(n) = o(g(n))$, $n^{\log n} = o((\log n)^n)$
\section{$(\log n)^n = o(n!)$}
Here we can use the ratio test to show that $\frac{f(n)}{g(n)}$ is monotonically decreasing as $n$ grows and, therefore, converges to zero as $n$ goes to infinity.
\newline
\newline
We want to show that: $\lim_{n\to\infty} \frac{\frac{f(n+1)}{g(n+1)}}{\frac{f(n)}{g(n)}} < 1$.
$= \lim_{n\to\infty} \frac{\frac{(\log n+1)^{n+1}}{(n+1)!}}{\frac{(\log n)^n}{n!}} = \lim_{n\to\infty} \frac{(\log n+1)^{n+1}}{(\log n)^n n}$
\newline
$\approx \lim_{n\to\infty} \frac{(\log n)^{n+1}}{(\log n)^n n} = \lim_{n\to\infty} \frac{(\log n)}{n} = 0 < 1$ (using part $3$ from above)
\newline
\newline
So we have:
\newline
$\lim_{n\to\infty} \frac{(\log n)^n}{n!} = 0$ and, therefore,
\newline
by the definition of $f(n) = o(g(n))$, $(\log n)^n = o(n!)$. 
% end Solution 

\end{document}
