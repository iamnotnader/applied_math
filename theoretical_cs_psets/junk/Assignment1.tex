\documentclass[11pt]{article}
\usepackage{slashbox}

\begin{document}
\title{COS340 Homework 1}
\author{Nader Al-Naji}
\date{\today}
\maketitle
-
\newline
\textbf{Problem 1}
\newline
\newline
Let $S = \{1,2, ...,n\}$. How many ordered pairs $(A,B)$ of subsets of S are there that satisfy $A \subseteq B$?
\newline
\newline
Let $B_k$ consist of the subsets of $S$ that contain exactly $k$ elements. If we fix $k$, then $|B_k| = $$n \choose k$ since there are
$n \choose k$ different ways to choose $k$ elements from a set of $n$. Now, each of these $k$ element subsets of $S$ has $2^k$ subsets of its
own, a fact we showed in class. So if we wish to enumerate all tuples $(A_{j}, B_{k,i})$ such that $A_{j} \subseteq B_{k,i}$ for some set $A_j$ with $B_{k,i} \in B_k$, then
there will be $2^k \times $$n \choose k$ of them since there are $n \choose k$ elements in the set $B_k$ and each one has
$2^k$ subsets. We get the quantity we want by summing over $k$. $\sum_{k=0}^n 2^n \times $$n \choose k$$ = $$\sum_{k=0}^n 2^n \times 1^{n-k} $$n \choose k$
$ = $$3^n$ tuples of $(A,B)$ such that $A, B \subseteq S$ and $A \subseteq B$.
\newline
\newline
Another way to think about this is to consider the the set of all strings of length $n$ having the numbers $0$, $1$, or $2$ as digits. By the 
generalized product rule, there are $3 \time 3 \times ... \times 3$ = $3^n$ of these strings. Now, if we think of each digit in one of these
strings as representing an element of the set $S$ and having it take on the digit $0$ as meaning the element is in set $S$ but not in
sets $A or B$, the digit $1$ as representing the element being in $B$ but not in $A$, and the digit $2$ as representing the element being
in both $A and B$, then we see that these strings have a one-to-one mapping to tuples $(A, B)$ from above and, by the bijection rule,
that there are $3^n$ of these tuples.
\newline
\textbf{Problem 2}
\newline
Prove that given $n > 0$ and a set of $n+1$ positive integers, none of them exceeding $2n$, there is at least one integer in this set that divides 
another integer in this set.
\newline
\newline
First, note that all even integers can be written in the form $k \times 2^b$ for some b where k is an odd number smaller than the integer itself and $b >= 0$. To see this, note that a factor that is even can always be made odd by pulling out a two. So consider all pairs of the form $(k, ak)$ where $k$ is an odd number and $a$ has the form $2^b$ for some $b \geq 0$. In a set of 2n numbers with $n > 0$, there are
n different values that k can take on, one for each of the odd numbers in the set of numbers $[0,2n]$. Further, note that every number in the set $[0,2n]$ belongs to exactly one of
these n different groups: a number is either odd, in which case it has its own group where it is the leftmost part of the tuple, or a number is even, in which case it can be broken down
into a multiple of $2^b$ and a distinct odd number $k$ and put into the $k^{th}$ group as the rightmost part of the tuple. Finally, note that if two numbers are in the same
group, then one of them divides the other. Since there are only $n$ different groups in a set of $2n$ numbers, then by the pigeonhole principle, selecting a set of $n+1$ numbers from
this set results in at least one collision with two of the numbers being in the same group and, therefore, one number dividing another. 
\newline
\textbf{Problem 3}
\newline
A chocolate bar consists of $n \times m$ (where $n,m \geq 1$) square pieces arranged in a rectangle. The square in the upper left corner is poisoned. Two players (Alice and Bob)
are taking turns breaking the bar into two rectangular bars along the lines, and eating the non-poisoned part. The player left with the lone poisoned piece loses the game.
Alice makes the first move. 
\newline
(a) For each pair $1 \leq n,m \leq 5$ which player has the winning strategy?
\newline
\newline
\begin{tabular}{r|ccccc}
\backslashbox[10pt][l]{n}{m} & 1 & 2 & 3 & 4 & 5\\\hline
1 & Bob & Alice & Alice & Alice & Alice\\
2 & Alice & Bob & Alice & Alice & Alice\\
3 & Alice & Alice & Bob & Alice & Alice\\
4 &	Alice	&	Alice  & Alice		& Bob & Alice\\
5 &	Alice	&	Alice  & Alice		& Alice& Bob
\end{tabular} 
\newline
\newline
The $1 \times 1$ case is trivial: Alice loses automatically. With any $n \times 1$ bar, Alice can turn the bar into a $1 \times 1$ and force Bob to lose so she 
wins the rest. If the bar is a $2 \times 2$, then all of Alice's move result in her losing since Bob can immediately turn the bar into a $1 \times 1$. With all the other $n \times 2$ moves, Alice can turn the bar into 
a square and force Bob to lose. Alice wins the $3 \times 1$ and $3 \times 2$ case by symmetry. With the $3 \times 3$ case, again all of her moves result in an 
automatic loss since Bob can immediately turn it into a $2 \times 2$ or $1 \times 1$ and force a loss. With the rest of the $n \times 3$, Alice can turn the bar
into a $3 \times 3$ for Bob and force him to lose. The first $3$ of the $n \times 4$ case come by symmetry again and $4 \times 4$ results in a loss since Bob can
immediately turn the bar into a $k \times k$ and force Alice to lose. Alice wins the $4 \times 5$ bar by turning it into a $4 \times 4$ for Bob. Finally, the
first $4$ of the $n \times 5$ case come by symmetry. In the $5 \times 5$ case Alice loses because Bob can turn the bar into a square after her turn, which makes
her lose.
\newline
\newline
(b) Who has the winning strategy for arbitrary $n,m \geq 1$? Prove your answer.
\newline
\newline
\textbf{Problem 4}

\end{document}