% Use this template to write your solutions

\documentclass[12pt]{article}

% Set the margins
%
\setlength{\textheight}{8.5in}
\setlength{\headheight}{.25in}
\setlength{\headsep}{.25in}
\setlength{\topmargin}{0in}
\setlength{\textwidth}{6.5in}
\setlength{\oddsidemargin}{0in}
\setlength{\evensidemargin}{0in}

% Macros
\newcommand{\myN}{\hbox{N\hspace*{-.9em}I\hspace*{.4em}}}
\newcommand{\myZ}{\hbox{Z}^+}
\newcommand{\myR}{\hbox{R}}

\newcommand{\myfunction}[3]
{${#1} : {#2} \rightarrow {#3}$ }

\newcommand{\myzrfunction}[1]
{\myfunction{#1}{{\myZ}}{{\myR}}}


% Formating Macros

\newcommand{\myheader}[4]
{\vspace*{-0.5in}
\noindent
{#1} \hfill {#3}

\noindent
{#2} \hfill {#4}

\noindent
\rule[8pt]{\textwidth}{1pt}

\vspace{1ex} 
}  % end \myheader 

\newcommand{\myalgsheader}[0]
{\myheader
{ {\bf{COS 340}} }
{ {\bf{Spring 2012}} }
{ {\bf{Collaborator 1}} : Diao, Lawrence }
{ {\bf{Collaborator 2}} : last name, first name}
}

% Running head (goes at top of each page, beginning with page 2.
% Must precede by \pagestyle{myheadings}.
\newcommand{\myrunninghead}[2]
{\markright{{\it {#1}, {#2}}}}

\newcommand{\myrunningalgshead}[2]
{\myrunninghead{COS 340 }{{#1}}}

\newcommand{\myrunninghwhead}[2]
{\myrunningalgshead{Solution to HW {#1}, Problem {#2}}}

\newcommand{\mytitle}[1]
{\begin{center}
{\large {\bf {#1}}}
\end{center}}

\newcommand{\myhwtitle}[3]
{\begin{center}
{\large {\bf Solution to HW {#1}, Problem {#2}}}\\
\medskip 
{\it {#3}} % Name goes here
\end{center}}

\newcommand{\mysection}[1]
{\noindent {\bf {#1}}}

%%%%%% Begin document with header and title %%%%%%%%%%%%%%%%%%%%%%%%%

\begin{document}

\myalgsheader

\pagestyle{plain}

\myhwtitle{2}{1}{Al-Naji, Nader}
% Example : \myhwtitle{1}{4}{Your name here}

\bigskip

% begin Solution 
There is a class of $n \geq 9$ students.
\newline
1. In how many ways can you form three groups of three students each ?
\newline
\newline
If we start by taking the ordering of the groupings and the ordering of the students within
the groupings to be important, then by the generalized product rule we have 
\newline
$(n)(n-1)(n-2) \times (n-3)(n-4)(n-5) \times (n-6)(n-7)(n-8)$
\newline
possibilities. However, because the ordering of each student within a group doesn't matter,
that is, because groups in which order matters form a k-to-one  mapping with $k = 3\times 2\times 1 = 3!$ onto groups where
order doesn't matter, we must divide each of the three factors by $3!$ by the division rule and get: 
\newline
\newline
$\frac{(n)(n-1)(n-2)(n-3)(n-4)(n-5)(n-6)(n-7)(n-8)}{3!3!3!}
= \frac{n!}{3!3!3!(n-9)!}$.
\newline
\newline
Finally, because the order of individual groups doesn't matter, that is because groups of three where
each group is distinct form a k-to-one mapping with $k = 3\times 2\times 1 = 3!$ onto groups where
each group is not distinct, we must again divide out a factor of 3! by the division rule and get:
\newline
\newline
$\frac{n!}{3!3!3!3!(n-9)!}$.
\newline
\newline

Another way to look at this is that we are counting the number of ways to choose three distinct groups
of three and a group of $n-9$ from a set of $n$ elements. Thus, using the multinomial theorem
derived in class, we have directly:
\newline
\newline
$n \choose {3, 3, 3, n-9}$ $ = \frac{n!}{3!3!3!(n-9)!}$
\newline
\newline
Then, as before, we must divide out a factor of $3!$ because groups of three where
each group is distinct form a k-to-one mapping with $k = 3\times 2\times 1 = 3!$ onto groups where
each group is not distinct and get as before:
\newline
\newline
$\frac{n!}{3!3!3!3!(n-9)!}$.
\newline
\newline

2. In how many ways can you form three groups, one with two students, one with three
students and one with four students 
\newline
\newline
This follows in the same way as the multinomial argument above but we will be rigorous and
begin by saying that order matters in each set. Then, the number of possibilities by the generalized product rule is:
\newline
$(n)(n-1) \times (n-2)(n-3)(n-4) \times (n-5)(n-6)(n-7)(n-8)$.
\newline
However, because the ordering of each student within each group doesn't matter,
that is, because we have a k-to-1 mapping from the ordered groups of students to the unordered
groups of students in each case, we must divide each term by its respective k. Thus, because 
each set is distinct unlike in problem $1$, we have directly:
\newline
\newline
$\frac{(n)(n-1)}{2!}\times\frac{(n-2)(n-3)(n-4)}{3!}\times\frac{(n-5)(n-6)(n-7)(n-8)}{4!} = \frac{n!}{2!3!4!(n-9)!}$.
\newline
\newline
Again, we could also have taken a straightforward multinomial approach and observed that
this is simply choosing sets of size 2, 3, 4, and $n-9$ from a set of size n. The number 
of ways one can do this, by the multinomial theorem is thus:
\newline
\newline
$n \choose {2, 3, 4, n-9}$ $ = \frac{n!}{2!3!4!(n-9)!}$.
\newline
\newline
And again no extra correction is needed because each set is distinct in this case.
% end Solution 

\pagebreak
\myalgsheader
\pagestyle{plain}
\myhwtitle{2}{2}{Al-Naji, Nader}
\bigskip
% begin Solution
Evaluate the following sums:
\newline
\newline
1. $\sum\limits_{i=0}^{n-1} $$n \choose i$ $n \choose {i+1}$
\newline
\newline
$\sum\limits_{i=0}^{n-1} $$n \choose i$ $n \choose {i+1}$$ = \sum\limits_{i=0}^{n-1} $$n \choose i$ $n \choose {n-1-i}$
\newline
The identity used to get from the first sum to the second is simply that $n\choose i$$=$$n \choose {n-i}$
which can be trivially shown (and was shown in precept). Now we can make a story for this new sum. It's 
as if we partition a bin of size $2n$ into two bins of size $n$ and count the number of ways we can have
$n-1$ balls distributed between these two bins of size n. We count the number of ways bin 1 can have $0$
balls and bin two can have $n-1$ balls, the number of ways bin 1 can have $1$ ball and bin 2 can have $n-2$ balls
and so on up to bin 1 having $n-1$ balls and bin 2 having $0$ balls. But this is simply the number of 
ways to distribute $n-1$ balls into a bin of size $2n$. Thus, this sum is equal to $2n \choose {n-1}$.
\newline
\newline
2. $\sum\limits_{0 \leq i,j,k \leq n \colon i+j+k = 2n}^{}$$n\choose i$$n \choose j$$n \choose k$
\newline
\newline
We can immediately put a story onto this sum. It is as if we took a bin of size $3n$ and broke it up
into three bins of size $n$ and then counted the number of ways we could distribute $2n$ balls between
these three bins of size $n$. But, as above, this is the same as if we took a bin of size $3n$ and  directly
counted the number of ways we could put $2n$ balls into it. So this sum is simply $3n \choose 2n$.
% end Solution


\pagebreak
\myalgsheader
\pagestyle{plain}
\myhwtitle{2}{3}{Al-Naji, Nader}
\bigskip
% begin Solution
Suppose there are $2$ boxes each containing 200 balls: 100 red balls numbered from 1 to 100,
and 100 blue balls numbered from 1 to 100. We pick one ball at random from each box. 
\newline
\newline
1. Given that at least one of the two balls picked is red, what is the probability that both balls are red?
\newline
\newline
The overall sample space is all permutations of the color and number of the first ball combined
with all permutations of the color and number of the second ball to make a total of 
$2\times 100 \times 2 \times 100 = 40000$ distinct events.
\newline
\newline
Let $O$ be the event that at least one of the two balls picked is red and let $B$ be the event that
both balls are red. We are picking two balls so let us enumerate the possibilities only
with respect to color for each pick:
\newline
1) First pick: Red; Second pick: Red
\newline
2) First pick: Red;  Second pick: Blue
\newline
3) First pick: Blue; Second pick: Red
\newline
4) First pick: Blue; Second pick: Blue
\newline
\newline
The probabilites of any of these four events happening is $\frac{100}{200}\times \frac{100}{200} = \frac{1}{4}.$
Event $O$ consists of events $1, 2,$ and $3$ so $P(O) = \frac{3}{4}.$ Event $B$ consists of event $1$ only so
$P(B) = \frac{1}{4}$. Finally, $O \cap B$ consists of event $1$ only and so $P(O \cap B) = \frac{1}{4}$ as well.
Using the definition of conditional probability we have:
\newline
\newline
$P(B | O) = \frac{P(B \cap O)}{P(O)} = \frac{\frac{1}{4}}{\frac{3}{4}} = \frac{1}{3}$
\newline
\newline
2. Given that at least one of the two balls is red and numbered 13, what is the probability that
both balls are red?
\newline
\newline
Again take the overall sample space as all permutations of the color and number of the first ball combined
with all permutations of the color and number of the second ball to make a total of 
$2\times 100 \times 2 \times 100 = 40000$ distinct events.
\newline
\newline
Now let $O_{r13}$ be the event that at least one of the balls is red and numbered thirteen and $B$ be the event that
both balls are red. We have a similar expression as in part one:
\newline
\newline
$P(B | O_{r13}) = \frac{P(B \cap O_{r13})}{P(O_{r13})}$
\newline
\newline
$P(O_{r13})$ is not difficult to compute; it is simply $1 - P($neither ball is red and 13$) 
= 1 - \frac{199}{200} \times \frac{199}{200} = \frac{399}{40000}$. Computing $P(B \cap O_{r13})$
is less trivial and we will do it by breaking the probability into three separate mutually exclusive events. 
\newline
\newline
A = First ball is red and not thirteen and the second ball is red and thirteen.
\newline
B = First ball is red and thirteen and the second ball is red and not thirteen.
\newline
C = The first and second ball are both red and numbered thirteen.
\newline
\newline
It should be clear that these three events are mutually exclusive and that these
three events cover $P(B \cap O_{r13})$.
Thus, $P(B \cap O_{r13}) = P(A) + P(B) + P(C) 
= \frac{99}{200}\times \frac{1}{200} + \frac{1}{200}\times \frac{99}{200} + 
\frac{1}{200} \times \frac{1}{200} = \frac{199}{40000}$. Now that we have both
$P(B \cap O_{r13})$ and $P(O_{r13})$, we can compute $P(B | O_{r13})$ as follows:
\newline
\newline
$P(B | O_{r13}) = \frac{P(B \cap O_{r13})}{P(O_{r13})} = \frac{\frac{199}{40000}}{\frac{399}{40000}}
= \frac{199}{399}$.
% end Solution

\pagebreak
\myalgsheader
\pagestyle{plain}
\myhwtitle{2}{4}{Al-Naji, Nader}
\bigskip
% begin Solution
Suppose we have a biased coin that turns up heads with probability $p$.
\newline
\newline
1. Calculate the probability that the number of heads in $n$ independent tosses of this coin is exactly $i$.
\newline
\newline
The probability of getting exactly $i$ heads and exactly $n-i$ tails in a specific order is simply
$p^i (1-p)^{n-i}$. There are $n \choose i$ different orders in which we can get $i$ heads 
in $n$ flips so the probability of getting $i$ heads in $n$ flips in any order is simply:
\newline
\newline
$P($exactly i heads$) = P($getting exactly i heads in a specific order$) \times $number of was to get i heads in n flips
$= $ $n \choose i$ $p^i  (1-p)^{n-i}$
\newline
\newline
2. Give a closed form expression for the probability that the number of heads in $n$ independent tosses of
this coin is odd.
\newline
\newline
Let $Odd(n)$ be a function that gives the probability that the number of heads is odd as a function of n.
Now, we can decompose $Odd(n)$ into two parts as follows:
\newline
\newline
$Odd(n) = (1-p) \times Odd(n-1) + p \times Even(n-1) = (1-p) \times Odd(n-1) + p \times (1 - Odd(n-1))$
\newline
\newline
where $Even(n)$ is a function that returns the probability the number of coin flips is even. Note that
$Even(n) = 1 - Odd(n)$ and that we've broken $Odd(n)$ up logically into the probability that the number
of heads after $n-1$ flips is even and the probability that the number of heads after $n-1$ flips is odd. If
the number of flips is even, then we need a heads on the $n^{th}$ flip; otherwise, we need a tails. If we
continue to expand in this way, we begin to notice a pattern:
\newline
\newline
$Odd(n) = (1-p) \times Odd(n-1) + p \times (1 - Odd(n-1)) = Odd(n-1)(1-2p)+p$
\newline
\newline
$= (Odd(n-2)(1-2p) + p)(1-2p) + p = Odd(n-2)(1-2p)^2 + p(1-2p) + p$
\newline
\newline
$= (Odd(n-3)(1-2p) + p)(1-2p)^2 + p(1-2p) + p = Odd(n-3)(1-2p)^3 + p(1-2p)^2 + p(1-2p) + p$
\newline
\newline
...
\newline
\newline
$= Odd(0)(1-2p)^n + p \sum\limits_{i=0}^{n-1} (1-2p)^i = p \sum\limits_{i=0}^{n-1} (1-2p)^i$ because $Odd(0) = 0$
\newline
\newline
$= p \frac{1-(1-2p)^n}{1-(1-2p)} = p \frac{1-(1-2p)^n}{2p} = \frac{1}{2} (1 - (1-2p)^n)$.
\newline
\newline
Note the geometric series observed at the ellipses and the closed form expression for the geometric series 
used on the last line.


% end Solution




\end{document}
